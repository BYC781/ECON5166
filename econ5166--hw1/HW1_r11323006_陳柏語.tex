\documentclass[11pt, oneside]{article}   	% use "amsart" instead of "article" for AMSLaTeX format
\usepackage{geometry}                		% See geometry.pdf to learn the layout options. There are lots.
\geometry{letterpaper}                   		% ... or a4paper or a5paper or ... 
\usepackage[parfill]{parskip}    			% Activate to begin paragraphs with an empty line rather than an indent
\usepackage{graphicx}				% Use pdf, png, jpg, or eps§ with pdflatex; use eps in DVI mode
								% TeX will automatically convert eps --> pdf in pdflatex		
\usepackage{bbm}
\usepackage{amssymb}
\usepackage{mathtools}
\usepackage{enumerate}
\usepackage{tikz}
\usepackage{graphicx}
\usepackage{CJK}

\usetikzlibrary{arrows}

\def\firstcircle{(90:1.75cm) circle (2.5cm)}
\def\secondcircle{(210:1.75cm) circle (2.5cm)}
\def\thirdcircle{(330:1.75cm) circle (2.5cm)}

%SetFonts

%SetFonts


\title{Data Science and Social Inquiry: HW1}
\author{
	R11323006 Bo-Yu, Chen
}
\date{\today}							

\begin{document}

\maketitle

\paragraph{}
(a) (2pts) What is $\hat{F}_n(4)?$ How about $\hat{F}_n(-3)$?
% Add a random fraction.
%%%%%%%%%%%%%%Your answer starts here%%%%%%%%%%%%%%%%%
\begin{align*}
\hat{F}_n(4) = 0.6\\
\hat{F}_n(-3) = 0.1
\end{align*}



%%%%%%%%%%%%%%%%%%%%%%%%%%%%%%%%%%%%%%%%%%%%%%%%%%%%%%

\paragraph{}
(b) (1pt) In the previous part, we find $\hat{F}_n(x)$ at two points, 
namely $x=4$ and $x=-3$. We can of course keep going and try other values of $x$, 
but this is rather repetitive and boring. 
Luckily, we have computers, and we know how to write program. 
Write a program to find out 
$\hat{F}_n(x)$ for $x = -10, \;-9.99, \;-9.98, ...., \;9.99, \;10$.

% Use \begin{align*} and \end{align*}.
%%%%%%%%%%%%%%Your answer starts here%%%%%%%%%%%%%%%%%



%%%%%%%%%%%%%%%%%%%%%%%%%%%%%%%%%%%%%%%%%%%%%%%%%%%%%%

\paragraph{}
(c) (1pt) Use the result from (b) and plot $\hat{F}_n(\cdot)$. 
How does it look like? Is it non-decreasing?
\footnote{$\hat{F}_n(\cdot)$ is non-decreasing if $\hat{F}_n(x_1) \leq \hat{F}_n(x_2)$ for $x_1 \leq x_2 $.} 

% Expectation.
%%%%%%%%%%%%%%Your answer starts here%%%%%%%%%%%%%%%%%
\begin{itemize}
    \item[] 
    Yes, it's non-decreasing.
\end{itemize}

\begin{figure}[htbp]
	\centering
	% uncomment below to add picture
	\includegraphics[width = 0.5\textwidth]{1(b)(c).png}
\end{figure}

%%%%%%%%%%%%%%%%%%%%%%%%%%%%%%%%%%%%%%%%%%%%%%%%%%%%%%

Now, let's investigate the statistical property of $\hat{F}_n(\cdot)$. 
For parts (d) - (f), we no longer assume $n=10$, and we will treat $X_1, X_2, ..., X_n$ as random. 
For simplicity, we focus on in $F_n(0)$, 
the probability that $X$ is less than or equal to $0$, 
for the rest of this exercise.

\paragraph{}
(d) (1pt) What is the expected value of $\hat{F}_n(0)$? Does it depend on $n$?
		  \textbf{Hint}: $\mathbbm{1}_{(-\infty,x]}(X_i)$ takes value only in 0 and 1. 
		  Which family of random variable only takes value in 0 and 1? 
		  What is its expected value? P.S. Your answer can be is related to $F_X(\cdot)$.

% Add a table.
%%%%%%%%%%%%%%Your answer starts here%%%%%%%%%%%%%%%%%
\begin{itemize}
    \item[]
    let
    \[
    Y_i = \mathbbm{1}_{(-\infty,x]}(X_i) =
    \begin{cases}
    1, &X_i \leq 0\\
    0, &\mbox{o.w.}
    \end{cases}
    \]
We can denote that
\[
     Y_i \overset{i.i.d.}{\sim} Bernoulli(p) \mbox{ where } p = P(X_i \leq 0)= F_X(0)
\]

so
\begin{align*}
\mathbb{E}(\hat{F}_n(0))&=\mathbb{E}(\frac{1}{n}\sum_{i=1}^{n}\mathbbm{1}_{(-\infty,x]}(X_i))\\
&=\frac{1}{n}\sum_{i=1}^{n}\mathbb{E}(\mathbbm{1}_{(-\infty,x]}(X_i))\\
&=\mathbb{E}(\mathbbm{1}_{(-\infty,x]}(X_i))\\
&=\mathbb{E}(Y)\\
&=P(Y=1) =  P(X_i \leq x)\\
&=F_X(x)
\end{align*}
\end{itemize}
%%%%%%%%%%%%%%%%%%%%%%%%%%%%%%%%%%%%%%%%%%%%%%%%%%%%%

\paragraph{}
(e) (1pt) What is the variance of $\hat{F}_n(0)$? Does it depend on $n$?

% Add a random figure.
%%%%%%%%%%%%%%Your answer starts here%%%%%%%%%%%%%%%%%

\begin{align*}
Var(\hat{F}_n(0)) &= Var(\frac{1}{n}\sum_{i=1}^n Y_i)\\
&= \frac{1}{n^2}\sum_{i=1}^n Var(Y_i)\\
&= np(1-p)\\
&= \frac{1}{n}F_X(0)[1-F_X(0)]\\
\end{align*}

\begin{itemize}
    \item[] The third equation holds because $Y_i$ is $i.i.d$
\end{itemize}


%%%%%%%%%%%%%%%%%%%%%%%%%%%%%%%%%%%%%%%%%%%%%%%%%%%%%%

\paragraph{}
(f) (1pt) What happens when $n\rightarrow\infty$? 
Do you think $\hat{F}_n(0)$ is a good estimator of $F_X(0)$?
\begin{itemize}
    \item[]
    By WLLN, 
    \begin{center}
         $\bar{X}\rightarrow \mathbb{E}(X)$, where $\bar{X} = \frac{1}{n}\sum_{i=1}^n X_i$ \\
    \end{center}
    and we know 
    
    \[
    \hat{F}_n(0)=\frac{1}{n}\sum_{i=1}^n\mathbbm{1}_{(-\infty,x]}(X_i), 
    \]
    
    so 
    \begin{center}
        $\hat{F}_n(0)\rightarrow \mathbb{E}(\mathbbm{1}_{(-\infty,x]}(X_i)) = F_X(0)$ 
        as
        $n \rightarrow \infty$
    \end{center}

    Therefore, $\hat{F}_n(0)$ is a good estimator of  $F_X(0)$ because it satisfies unbiasedness and consistency.
\end{itemize}

An alternative way to study the statistical property of $\hat{F}_x(0)$ is through conducting simulation experiments, which are commonly known as \textbf{\textit{Monte Carlo simulations}}. 

A simulation experiment typically contains many rounds. 
In each round, we will draw a random sample $(X_1, X_2, ..., X_n)$ from a 
distribution chosen by the researcher and calculate $\hat{F}_n(x)$ 
given $(X_1, X_2, ..., X_n)$. For example, we can set $n=100$, generate
\[
X_1, X_2, ..., X_n \overset{i.i.d.}{\sim} N(0,1), 
\]
in each round of the simulation, and calculate the resulting $\hat{F}_n(0)$.


If we run $B = 10,000$ rounds, we will get $10,000$ realizations of $\hat{F}_n(0)$. We then evaluate the performance of $\hat{F}_n(0)$ by comparing $1000$ realizations of $\hat{F}_n(0)$ to its true value $F_X(0).$

\paragraph{}
(g) (1pt) What is $F_X(0)$, the true value of the parameter of interest, 
given that $X \sim N(0,1)$?
\[
F_X(0) = P(X<0) = \frac{1}{2}
\]

\paragraph{}
(h) (1pt) Set the seed with \texttt{numpy.random.seed(5516)},
 use \texttt{numpy.random.normal} to generate $X_1, X_2, ..., X_n \overset{i.i.d.}{\sim} N(0,1)$ 
 for $n=100$, and calculate $\hat{F}_{n,1}(0)$, 
 where the subscript $1$ means that $\hat{F}_{n,1}(0)$ 
 is obtained in the first round of simulation. 
 Repeat $10,000$ times and collect the estimates $\hat{F}_{n,1}(0)$, $\hat{F}_{n,2}(0)$, 
 ..., and $\hat{F}_{n,10000}(0)$. 
 Calculate the \textbf{mean squared error} (MSE)

\[
\frac{1}{10000}\sum_{b=1}^{10000} \big[\hat{F}_{n,b}(0) - F_X(0)\big] ^ 2,
\]

which is the average squared distance between the estimator $\hat{F}_{n,b}(0)$ and its true value $F_X(0)$.

\paragraph{}
(i) (1pt) Repeat (h) with $n=200$ and $n=500$. Is MSE larger or smaller when $n$ is larger?
\begin{itemize}
    \item[] Ans: MSE is smaller when n is larger.
\end{itemize}

% Summarize your answer in (i) \& (j) in the table below
\begin{table}[htbp]
	\centering\begin{tabular}{|c|c|c|c|}
	\hline
	n = & 100 & 200 & 500 \\ \hline
	MSE & 0.0025   & 0.00124   & 0.0005  \\ \hline
	\end{tabular}
	\caption{MSE vs. sample size}
	\label{<label>}
\end{table}



\paragraph{}
(j) \textbf{(Bonus, 2pts)} the \textbf{Central Limit Theorem} (CLT) implies that 
\[
\sqrt{n}(\hat{F}_{n}(0) - F_X(0)) 
\]
will converge to the normal distribution. We can verify that CLT holds in our case by plotting the histogram of 
\[
\sqrt{n}(\hat{F}_{n,b}(0) - F(0)), \; b = 1,2, ...., 10000
\]
for $n=500$. Does your plot support CLT?

% No need to include your code. Just the plot is enough.
\begin{itemize}
    \item [] The orange curve is the p.d.f. of $N(0,\frac{1}{4})$ because by CLT, we have
    \[
    \sqrt{n}(\hat{F}_{n}(0) - F_X(0)) \sim N(0, F_X(0)[1-F_X(0)]) = N(0, \frac{1}{4})
    \]
    and we can find that the histogram is quite similar to the p.d.f. curve, so I think we have verified that CLT holds
\end{itemize}
\begin{figure}[htbp]
	\centering
	% uncomment below to add picture
	\includegraphics[width = 0.7\textwidth]{CLT_image.png}
	\label{fig:CLT}
\end{figure}

\end{document}  

