\documentclass[11pt, oneside]{exam}   	% use "amsart" instead of "article" for AMSLaTeX format
\usepackage{geometry}                		% See geometry.pdf to learn the layout options. There are lots.
\geometry{letterpaper}                   		% ... or a4paper or a5paper or ... 
%\geometry{landscape}                		% Activate for rotated page geometry
\usepackage[parfill]{parskip}    			% Activate to begin paragraphs with an empty line rather than an indent
\usepackage{graphicx}				% Use pdf, png, jpg, or eps§ with pdflatex; use eps in DVI mode
								% TeX will automatically convert eps --1 pdf in pdflatex		
\usepackage{bbm}
\usepackage{amssymb,amsmath,amsthm,bm,pgfplots}
\usepackage{mathtools}
\usepackage{relsize}
\usepackage[style=apa, backend=biber]{biblatex}
\addbibresource{reference.bib}
\usepackage{enumitem}
\usepackage{hyperref}
\usepackage{comment}

\usetikzlibrary{arrows}


\newcommand{\E}[1]{\mathbb{E}\left[#1\right]}


\printanswers

\title{Data Science and Social Inquiry: HW4}
\author{Yu-Chang Chen and Ming-Jen Lin}
\date{\today}							

\begin{document}

\maketitle
\section*{Question 1: Implementing Newton's Method}
In this question, we will implement Netwon's method, which is a specific version
of the gradient descent algorithm, to minimize the function
\[
	f(x) = 0.05x^4 + 0.1x^3 - 0.75x^2 - x + 3.
\]
Recall that the Newton's method uses Hessian as learning rate and iterates in the
following way
\[
	x_{k+1} = x_k - \frac{1}{f''(x_k)} \cdot f'(x_k).
\]
\begin{enumerate}[label = (\emph{\alph*})]
	\item (1 pt) Plot $f(x)$ in Python. Where is the global minimum?
	      \begin{solution}

	      \end{solution}
	\item (1 pt) Run the Newton's method with initial point $x_0 = 5$
	      and iterate 1,000 times. Plot the first 1,000 iterations on a graph.
		  Does it converge to the global minimizer?
	      \begin{solution}

	      \end{solution}
	\item (1 pt) Run the Newton's method with initial point $x_0 = -1$
	      and iterate 1,000 times. 
		  Plot the first 1,000 iterations on a graph.
		  Does it converge to the global minimizer?
	      \begin{solution}

	      \end{solution}
\end{enumerate}

\section*{Question 2: Apply Gradient Descent to MLE}
The maximum likelihood estimator (MLE) estimates a parameter using the maximizer
of the log-likelihood function. That is,
\[
	\hat{\theta}_{\text{MLE}} = \underset{\theta\in \mathbb{R}}{\arg\max}\; \frac{1}{n}\sum_{i=1}^n ln(f(x_i|\theta)),
\]
where $f(x_i|\theta)$ is the p.d.f. (or p.m.f.) that generates observations
$x_1, x_2,..., x_n$.

In the case of normal distribution with known variance $\sigma^2 = 1$,
the MLE of the location parameter $\mu$ is
\[
	\hat{\mu}_{\text{MLE}} = \underset{\theta\in \mathbb{R}}{\arg\max}\;
	\frac{1}{n}\sum_{i=1}^n\bigg[-\frac{1}{2}ln(2\pi) - \frac{1}{2}(x_i-\mu)^2) \bigg]
\]
Suppose that our observations $x_i$'s are 
\[
	3, 3, 2, 2, 4, 2, 4, 4, 3, 1.
\]
Answer the following questions.
\begin{enumerate}[resume*]
	\item (1 pt) Derive the Hessian of the objective function. Is it concave?
	      \footnote{For maximization, we prefer concave functions since local maximum must be global maximum for concave functions.}
	      \begin{solution}

	      \end{solution}
	\item (1 pt) Analytically solve $\hat{\mu}$ by the first order condition.
	      \begin{solution}

	      \end{solution}
	\item (1 pt) Use the Newton's method:
	      \[
		      x_{k+1} = x_k + \frac{1}{f''(x_k)} \cdot f'(x_k)\footnote{Notice that to maximize a function, we update in the direction of the gradient.}
	      \]
	      to solve $\hat{\mu}_{\text{MLE}}$ numerically. 
	      How many iterations does it take to find $\hat{\mu}$?
	      \begin{solution}

	      \end{solution}
\end{enumerate}

\section*{Question 3: Simulating Multiple Testing}
In this question, we will simulate 1,000 coins and flip each coin 100 times.
Our goal is to test whether each coin $i$ is fair or not:
\begin{align*}
	 & \mathcal{H}_{i,0}: \text{Coin } i \text{ is a fair coin,}      \\
	 & \mathcal{H}_{i,1}: \text{Coin } i \text{ is not a fair coin. }
\end{align*}
The purpose of this question is to demonstrate that, without adjustment for
multiple testing, classical testing procedure may result in lots of false discovery.
We'll start by constructing a ``single'' test for each coin.

Let $X_{i,1}, X_{i,2}, ..., X_{i,100}\overset{i.i.d.}{\sim}Bernouli(0.5)$ 
denote the 100 flips of coin $i$ and $\bar{X}_i=\frac{1}{100}\sum_{j=1}^{100}X_{i,j}$ be their average.
The Central Limit Theorem implies that
\[
	\bar{X}_i \overset{d}{\approx}\mathcal{N}\bigg(E[\bar{X}_i],\; Var(\bar{X}_i)\bigg),
\]
and we can use the normal approximation to construct a t-test that rejects $\mathcal{H}_{i,0}$ if
\[
	|\bar{X}_i - 0.5| > c.
\]
Let $\Phi(\cdot)$ be the cumulative distribution function of standard normal
distribution.
\begin{enumerate}[resume*]
	\item (1 pt) Calculate $E[\bar{X}_i]$ and $Var(\bar{X}_i)$.
	      \begin{solution}

	      \end{solution}
	\item If we would like the test to have size 0.05, what value of the decision
	      cutoff $c$ should we choose?
	      Hint: your answer will make use of $\Phi(\cdot)$.
	      \begin{solution}

	      \end{solution}
\end{enumerate}
Now, set \texttt{numpy.random.seed(13579)}
and use \texttt{numpy.random.binomial()} to generate 1,000 $\times$ flips
for 100 coins.
\begin{enumerate}[resume*]
	\item Apply the test you just constructed to test $\mathcal{H}_{i,0}$ for
	      coins. How many false discovery did you find?
	      \begin{solution}

	      \end{solution}
	\item Implement the Bonferroni correction for multiple testing. 
		  What is the decision cutoff for each hypothesis?
		  How many false discovery did you find? 
	      \begin{solution}

	      \end{solution}
\end{enumerate}
\end{document}

